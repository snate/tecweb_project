%%%%%%%%%%%%%%%%%%%%%%%%%%%%%%%%%%%%%%%%%
% Arsclassica Article
% LaTeX Template
% Version 1.1 (10/6/14)
%
% This template has been downloaded from:
% http://www.LaTeXTemplates.com
%
% Original author:
% Lorenzo Pantieri (http://www.lorenzopantieri.net) with extensive modifications by:
% Vel (vel@latextemplates.com)
%
% License:
% CC BY-NC-SA 3.0 (http://creativecommons.org/licenses/by-nc-sa/3.0/)
%
%%%%%%%%%%%%%%%%%%%%%%%%%%%%%%%%%%%%%%%%%

%----------------------------------------------------------------------------------------
%	PACKAGES AND OTHER DOCUMENT CONFIGURATIONS
%----------------------------------------------------------------------------------------

\documentclass[
article,
10pt, % Main document font size
oneside, % One page layout (no page indentation)
BCOR5mm, % Binding correction
]{scrartcl}
\usepackage[a4paper,margin=3cm]{geometry}
\usepackage[italian]{babel}
\usepackage[utf8]{inputenc}
\usepackage{atbegshi}% http://ctan.org/pkg/atbegshi
\usepackage{tikz}
\usetikzlibrary{shapes}
\usepackage{amsmath}
\usepackage{xspace}
\usepackage[usestackEOL]{stackengine}
\usepackage{graphicx}
\usepackage{caption}
\usepackage{subcaption}
\usepackage{float}
\newcommand{\A}{\ensuremath{\mathcal{A}}\xspace}
\newcommand{\B}{\ensuremath{\mathcal{B}}\xspace}
\newcommand\pa[1]{\ensuremath{\left(#1\right)}}
\AtBeginDocument{\AtBeginShipoutNext{\AtBeginShipoutDiscard}}

\hyphenation{Fortran hy-phen-ation} % Specify custom hyphenation points in words with dashes where you would like hyphenation to occur, or alternatively, don't put any dashes in a word to stop hyphenation altogether

%----------------------------------------------------------------------------------------
%	TITLE AND AUTHOR(S)
%----------------------------------------------------------------------------------------

\begin{document}

\begin{center}
\title{
  \begin{figure}[h!]
  \centering
  \includegraphics{images/icona_titolo.png}
  \end{figure}
  Valutazione dell'accessibilità del sito \\
  Libreria di Valyria
}
\author{Carlo Munarini 1050128 \\
Sebastiano Valle 1050123}
\end{center} % The article author(s) - author affiliations need to be specified in the AUTHOR AFFILIATIONS block






%\date{} % An optional date to appear under the author(s)

%----------------------------------------------------------------------------------------

%----------------------------------------------------------------------------------------
%	TABLE OF CONTENTS & LISTS OF FIGURES AND TABLES
%----------------------------------------------------------------------------------------

\maketitle % Print the title/author/date block

\setcounter{page}{1} % Set the depth of the table of contents to show sections and subsections only

{\raggedleft\vfill\Longstack[l]{%
Indirizzo: \texttt{http://tecnologie-web.studenti.math.unipd.it/tecweb/$\sim$aongaro} \\
Email referente: \texttt{sebastiano.valle@hotmail.it}
}
\raggedright{}
\newpage{}
\tableofcontents % Print the table of contents



\listoffigures % Print the list of figures

\listoftables % Print the list of tables

\section{Struttura}
Di seguito sono illustrate le modalità di progettazione della struttura delle
pagine all'interno del sito.

\subsection{Parti comuni a tutte le pagine}
Tutte le pagine sono state scritte seguendo lo standard \textit{XHTML 1.0 Strict} e come codifica è stata scelta UTF-8 dal momento che nel sito sono
presenti parole accentate.
Una sezione \textbf{head} ed una sezione \textbf{body} sono state inserite in
tutte le pagine con la stessa struttura.

\subsubsection{Head} %TODO @Graziano @Carlo
Sono presenti i seguenti tag nelle sezioni head di tutte le pagine:
\begin{itemize}
\item \textbf{title:} permette di visualizzare sulla finestra del browser il titolo della pagina visualizzata, dal particolare al generale
\item \textbf{meta title:} indica il titolo della pagina in un eventuale snippet, anch'esso dal particolare al generale
\item \textbf{meta description:} in questo tag viene inserita la breve descrizione della pagina visualizzata in un eventuale snippet
\item \textbf{meta author:} in questo tag sono indicati i componenti del gruppo
\item \textbf{meta keywords:} parole che aiutano un motore di ricerca a trovare la pagina grazie a dei termini di importanza focale
\item \textbf{meta robots:} tag che indica ad un eventuale spider se indicizzare la pagina e se seguire i link da essa uscenti
\item \textbf{meta keywords:} parole che aiutano un motore di ricerca a trovare la pagina grazie a dei termini di importanza focale
\item \textbf{meta reply-to:} %TODO@Graziano
\item \textbf{meta Classification:} tag che serve ad indicare l'argomento trattato dalle pagine del sito
\item \textbf{meta viewport:} %TODO@Carlo
\item \textbf{link shortcut icon:} icona visibile a fianco al titolo della scheda nel browser, aiuta a identificare meglio le schede di \textit{What To Visit} se un utente avesse più schede aperte nel suo browser
\item \textbf{link stylesheet:} collegamento a foglio di stile CSS %TODO@Carlo
\end{itemize}

\subsubsection{Body}
Affinchè l'utente si sentisse il meno disorientato possibile all'interno di \textit{What To Visit}, si è cercato di progettare il sito con un layout essenziale e che mettesse in primo piano il contenuto aspettato in tutte le pagine.
Sono presenti questi elementi strutturali nei corpi di tutte le pagine:
\begin{itemize}
\item un header, dove vi è il logo del sito;
\item un'ampia parte centrale, dove vengono visualizzati i contenuti richiesti dall'utente;
\item un footer, dove sono presenti link ed informazioni di poco rilievo e un'indicazione riguardo la validità della pagina.
\end{itemize}

\subsection{Homepage}
Come pagina principale del sito, si è pensato di esporre in primo piano all'utente la scelta delle tre categorie delle località.
A partire da queste l'utente può arrivare nelle pagine delle categorie, dove può trovare le liste delle località presenti in queste.

Dal momento che la homepage è l'unica pagina facilmente riconoscibile data la sua struttura con tre titoli di indirizzamento, la breadcrumb è stata omessa perchè si è assunto che gli utenti riuscissero a dedurre che si trovano nella homepage quando vi sono dentro (anche grazie all'URL).

Per poter comunque fornire collegamenti alle pagine che non sono di contenuto ma che sono significative (Chi Siamo e F.A.Q.), i link a queste sono stati inseriti nell'header della pagina a fianco del logo; in questo modo, anche se sono di importanza secondaria rispetto ai tre pannelli visualizzati nella pagina, rimangono comunque nella parte visibile del sito quando questo viene aperto (\texttt{http://en.wikipedia.org/wiki/Above\_the\_fold\#In\_web\_design}).

Nel footer, oltre alle indicazioni di validità della pagina, sono stati lasciati i link restanti alle pagine che non sono di contenuto.
 % 90%

\section{Schema organizzativo}\label{sec:schema-org}
Lo schema organizzativo del sito è chiaramente ambiguo:
\begin{itemize}
\item Non ci sono abbastanza informazioni per determinare se un libro nella
sezione ``nuovi" della home sia un libro uscito pubblicamente di recente o un
libro appena arrivato nella collezione della libreria;
\item Nella sezione ``più votati" nella home non è possibile sapere se
lo stesso libro comparirà in tale sezione in due visite differite da parte di
un utente;
\item Nella sezione ``Generi", un libro potrebbe appartenere a due o più
categorie diverse (ad esempio ci si aspettava di trovare \textit{Divergent} in
``Narrativa");
\item Nella sezione ``Generi", dopo la selezione del genere non è chiaro
l'ordine in cui sono visualizzati i libri;
\item La ricerca, per come è presentata, farebbe pensare ad uno schema
organizzativo ambiguo; tuttavia, si parlerà più in dettaglio di ciò nella
sezione seguente ed è doveroso notare che non sempre fornisce i risultati
attesi (ad esempio eseguendo la ricerca per autore e immettendo la lettera ``r"
viene mostrato solamente un libro di Glenn Cooper).
\end{itemize}

In particolare, la ricerca per collana è ambigua perchè non vi sono
indicazioni su come svolgere la ricerca: l'utente quindi non sa cosa può
inserire precisamente in tale campo per effettuare la ricerca.

\subsection{Elenchi di libri}\label{sec:schema-elenchi}
Dopo varie ricerche e dopo aver esplorato tutti i generi letterari offerti, ai
verificatori è parso che i libri in questi elenchi siano visualizzati secondo
l'ordine cronologico di apparizione nella libreria, dal più vecchio al
più recente. Tuttavia questa scelta sembra poco plausibile poichè si ritiene
che gli utenti siano interessati alle opere più nuove, come quelle proposte
nella home.

Per questo motivo si ritiene che manchi del tutto uno schema organizzativo e
che l'ordine trovato nella generazione degli elenchi sia \textbf{incidentale},
mentre sarebbe sufficiente uno schema organizzativo esatto che rispetta
l'ordine alfabetico (o presentando i libri dal più recente al meno recente,
evidenziando la data di pubblicazione/inserimento) per poter aiutare l'utente
nell'esplorazione del sito. % 90%

\section{Accessibilità}
Come standard di accessibilità è stato scelto di raggiungere il grado WAI-AA
WCAG 2.0. Di seguito sono descritte gli accorgimenti adottati dal nostro sito
per perseguire questo obiettivo.

\subsection{Trasformazione elegante}
Le pagine del nostro sito rimangono accessibili anche se l'utente non può
utilizzare alcune tecnologie o funzionalità per scelta personale, se il
suo dispositivo di navigazione non le supporta o per cause di forza maggiore
(e.g. svantaggi dal punto di vista fisico e psichico).

\subsubsection{Separazione struttura-comportamento}
In ogni pagina del sito non è stato introdotto alcun attributo per la gestione
degli eventi associata ad un elemento strutturale e gli script sono inclusi
nella pagina e non fanno parte del corpo di questa.
Gli script JavaScript, a loro volta, non si preoccupano di creare intere pagine
dinamicamente, ma si limitano a creare frammenti di queste in modo opportuno,
ovvero quando ciò \textbf{non} è possibile staticamente.

\subsubsection{Separazione struttura-presentazione}
Le pagine del sito non contengono alcun foglio di stile ma ogni foglio di stile
viene incluso in queste. Oltre a ciò, ogni elemento strutturale contiene
solamente attributi relativi al suo significato semantico e non presentazionale
(ad esempio non sono stati utilizzati gli attributi \texttt{bgcolor} e
\texttt{font}).

Ogni id e ogni classe è stata dichiarata utilizzando nomi relativi alla
semantica e non al modo in cui verranno presentati.

\subsubsection{Separazione presentazione-comportamento}
Gli script JavaScript non vanno a modificare le regole con cui gli elementi
sono presentati all'interno delle pagine del sito, ma si limitano ad attribuire
o aggiungere ad essi classi che verranno trattate con dei fogli di stile.

\subsubsection{Porzioni di sito visibili}
L'area cosidetta ``\textit{above the fold}" contiene sempre i contenuti di maggior
rilievo della pagina visualizzata, mentre elementi di minore importanza sono
presenti in parti come il footer che non sempre possono essere visibili all'apertura di una nuova finestra.

\subsection{Linee guida per l'accessibilità}
In questa sottosezione viene descritto come il gruppo ha rispettato le linee
guida del WAI.

\subsubsection{Alternative a contenuti audio e visivi}
Nel sito non sono presenti contenuti audio e video né applet, quindi non ci si
pone il problema per questi. Al contrario, ogni immagine ha un attributo
\texttt{alt} che fornisce l'equivalente testuale nel caso in cui non fosse
possibile visualizzare l'immagine.

\subsubsection{Non fare affidamento sul colore}
Sono stati utilizzati tool per la verifica dell'accessibilità che controllano
anche l'uso corretto del colore nel sito.

\subsubsection{Uso appropriato dei tag}
I tag sono stati utilizzati per il loro significato semantico, non sono state
utilizzate tabelle per definire il layout delle pagine e, togliendo i fogli di
stile, il sito rimane accessibile anche con le impostazioni predefinite dei
browser.

\subsubsection{Linguaggi naturali}
Sono stati utilizzati marcatori sia per la pronuncia di parole in lingua
straniera che per estendere la pronuncia di abbreviazioni e acronimi.

\subsubsection{Trasformazione elegante delle tabelle}
Non sono presenti tabelle nel sito.

\subsubsection{Trasformazione elegante delle nuove tecnologie}
\begin{enumerate}
\item CSS3%\subsubsection{CSS 3} %TODO con Carlo e Graziano
\item \textbf{JavaScript}: se un utente ha disattivato JavaScript sul proprio
browser non è in grado di visualizzare i commenti (parte secondaria di contenuto), non visualizza il placeholder nella barra di ricerca, quando clicca su un link ``esterno", questo viene aperto sulla stessa finestra e le immagini sulla home non fungono da link alle pagine delle categorie; tuttavia, senza queste funzionalità il sito continua ad offrire una più che buona user experience
\item Geolocalizzazione %TODO Graziano
\end{enumerate}

\subsubsection{Contenuti che cambiano nel corso del tempo}
\begin{enumerate}
\item la searchbar cambia dimensione quando assume il focus e viene
visualizzato un placeholder solamente se quando questa perde il focus non è
presente testo (altrimenti rimane l'input immesso dall'utente);
\item premendo sui pulsanti ``Visualizza commento",``Pubblica commento" e
``Nascondi commenti" vengono rispettivamente visualizzati i commenti presenti,
compare la form di inserimento commenti e vengono nascosti i commenti
precedentemente visualizzati.
\end{enumerate}

Questi elementi non causano problemi ad eventuali utenti che soffrono di
epilessia poichè il loro cambio di stato non è troppo rapido. Allo stesso
tempo, questi elementi si aggiornano dopo dei click e mantengono il loro stato
fino ad una successiva interazione tramite click non causando un senso di
disagio nell'utente che altrimenti vedrebbe il layout modificarsi di continuo
sotto i suoi occhi.

\subsubsection{Interfacce utente}
Non sono previste delle interfacce utenti quali comandi vocali ed access key
per il sito.

Al contrario, sono stati previsti dei tab index per:
\begin{enumerate}
\item poter saltare delle voci di navigazione ridondanti;
\item accedere con priorità ai link potenzialmente più interessanti per
l'utente (secondo previsioni dei componenti del gruppo).
\end{enumerate}

\subsubsection{Indipendenza da dispositivo}
Come detto nelle sezioni \ref{sec:fangs} e \ref{sec:lynx}, le pagine del sito
sono state provate rispettivamente anche con emulatori di screen reader e
browser testuali.

Non state utilizzate aree di immagini come link.

\subsubsection{Meccanismi di fallback}
Ogniqualvolta che una certa tecnologia non fosse disponibile per visualizzare i
contenuti previsti, si è scelto di non fornire il contenuto previsto (perché
di secondaria importanza) oppure il sito offre dei meccanismi per i quali la
degradazione è elegante (ad esempio la searchbar, se acquista il focus, si
ingrandisce istantaneamente anzichè effettuare una transizione).

\subsubsection{Raccomandazioni W3C}
Come detto in precedenza, le immagini presentano sempre l'alternativa testuale
inserita utilizzando la tecnologia offerta da W3C.

Non sono presenti formati come shockwave e PDF.

\subsubsection{Orientamento}
Vengono inoltre forniti una mappa del sito ed una pagina di F.A.Q. riferite in
ogni pagina del sito. Oltre a questo, in tutte le pagine eccetto la home (dove
non è ritenuta necessaria) è presente una breadcrumb che indica all'utente la
sua posizione all'interno del sito.

\subsubsection{Navigazione}
I link sono evidenziati in modo che siano distinguibili attraverso un test di \textit{Drue Miller} e forniscono sempre un attributo ``title" che informa
l'utente sul contenuto della destinazione.
Infatti:
\begin{itemize}
\item i link sono in grassetto se presenti nel testo;
\item i colori sono ben distinguibili come spiegato nella sezione \ref{sec:Pres-Colore};
\item i link visitati e non sono sempre riconoscibili gli uni dagli altri, ad
eccezione del caso in cui questi siano nel \textbf{nav} o nel \textbf{footer};
\item altri accorgimenti su come vengono trattati e specializzati gli elementi
della navigazione vengono discussi nella sezione \ref{sec:presentazione} a pagina \pageref{sec:presentazione}.
\end{itemize}

In tutte le pagine eccetto la home (dove la navigazione viene indirizzata
volutamente verso le categorie), è presente un menù di navigazione contenente
i ``sibling" della pagina o, se l'utente si trova in una pagina di una
località, visualizza i riferimenti alla homepage e alle categorie.
In questo elemento viene indicata anche la pagina corrente con un'icona
oppure, se l'utente è presente in una pagina do una località, viene
visualizzata un'icona di colore differente dalla precedente vicino alla
voce relativa alla categoria di appartenenza.

\subsubsection{Semplicità dei contenuti}
Il layout è coerente, consistente e riconoscibile in tutte le pagine del sito,
come descritto nelle sezioni \ref{sec:struttura} e \ref{sec:presentazione}.

Si è cercato di tenere un linguaggio semplice nei contenuti del sito.
 % 80%

\section{Esperienza utente}\label{sec:user-exp}
Di seguito vengono riportate le considerazioni riguardo la qualità del sito in
termini di \textbf{user experience}.

\subsection{Consistenza dell'interfaccia}
L'interfaccia è consistente e utilizza sempre gli stessi elementi grafici per
denotare elementi comuni a diverse pagine, eccezion fatta per:
\begin{itemize}
\item i link in \texttt{accesso.html} e \texttt{registrazione.html};
\item la breadcrumb (in tali pagine contrassegnata con l'id \texttt{path}) non
sempre è strutturata nello stesso modo, contenendo una o due voci a seconda
dei casi (e.g. in \texttt{about.html} e nella pagina generata dalla ricerca);
\item l'utente per poter inserire un commento (e quindi completare un task
relativo alla pagina dove è descritto un libro) deve accedere ad una nuova
pagina per scrivere il commento, senza che in questa nuova pagina siano
visualizzate le informazioni sul libro.
\end{itemize}

Sebbene presenti i problemi sopra descritti, l'interfaccia è comunque ritenuta
consistente rispetto al contesto in cui è inserita (una libreria) e alle
medie aspettative che un utente avrebbe visitando un sito di una libreria.

\subsection{Area visibile}
Ogni pagina ha un'ampia intestazione che occupa una buona parte di spazio;
tuttavia, non riducendo la pagina a dimensioni eccessivamente ridotte, il
contenuto principale rimane parzialmente visibile (se non questi, almeno i
titoli del contenuto).

Nelle descrizioni dei libri i commenti occupano una posizione di fondo.
Probabilmente i commenti dovrebbero trovare uno spazio nell'area visibile
(\textit{above the fold}). Se non venissero inseriti in questa zona della
pagina, servirebbe almeno un pulsante o un'ancora presente \textit{above the
fold} per poter raggiungere facilmente la sezione della pagina riservata ai
commenti.

\subsection{Informazioni utili}
L'utente non sempre dispone di informazioni accessorie ma utili durante la sua
navigazione: di seguito verranno discusse la loro presenza e, se presenti, le
modalità con cui queste sono state realizzate.

\subsubsection{Informazioni sulla propria posizione}
L'utente non dispone di sufficienti informazioni sulla barra contestuale
(\textit{path} o \textit{breadcrumb}) per capire in che modo ha raggiunto la
pagina in cui si trova: 

 % 70%

\section{Test effettuati} %TODO Graziano, Federica, Carlo i test effettuati da voi
In questa sezione si elencano i test effettuati per il sito in esame.
Per i test di accessibilità eseguiti in remoto, il sito è stato caricato su
GitHub Pages.

\subsection{Validazione XHTML} %DONE
Ogni pagina è stata validata con il tool offerto da W3C all'indirizzo
\texttt{http://validator.w3.org/\#validate\_by\_uri}, con il risultato che
le seguenti pagine non sono valide:
\begin{enumerate}
\item pagina principale del sito;
\item pagina descrittiva di un libro (generata tramite CGI).
\end{enumerate}

\subsection{Validazione CSS}
Ogni foglio di stile CSS è stato validato con il tool offerto da W3C
all'indirizzo
\texttt{https://jigsaw.w3.org/css-validator/\#validate\_by\_input}.

\subsection{Cynthia Says} %DONE
Uno dei tool che è stato utilizzato per l'accessibilità delle pagine è Cynthia
Says, inserendo l'URL di ogni pagina caricata su GitHub Pages e richiedendo la
conformità a WCAG 2.0 AAA. Questa scelta è dovuta al fatto che i tentativi per
verificare l'accessibilità di una pagina sono limitati a 10 nell'arco di 24
ore.
Di seguito vengono riportati i risultati ottenuti per \textbf{tutte} le pagine
del sito:

\begin{table}[h!]
\begin{center}
\begin{tabular}{ | l | c | }
  \hline
  Conformità & Esito \\
  \hline
  WCAG 2.0 A & Non conforme \\
  \hline
  WCAG 2.0 AA & Non conforme \\
  \hline
  WCAG 2.0 AAA & Non conforme \\
  \hline
\end{tabular}
\caption{Conformità allo standard WAI 2.0 X WCAG}
\end{center}
\end{table}

Le pagine del sito non sono conformi a WCAG 2.0 A, come detto nella sezione
\ref{sec:accessibilita}.

\subsection{WAVE}
WAVE (Web Accessibility Evaluation Tool) è uno strumento offerto da
\textit{WebAIM} (WEB Accessibility In Mind) per verificare l'accessibilità
delle pagine. Questo strumento analizza una pagina e fornisce errori,
avvertimenti (warnings), caratteristiche positive, struttura basilare di una
pagina e indicazioni sul contrasto.

Più in particolare, sono stati corretti gli errori trovati (relativi agli
standard 508 e WCAG) non relativi alla lingua delle pagine, dal momento che
nonostante il tag HTML fosse ben-formato e contenente l'indicazione sulla
lingua utilizzata nella pagina in modo corretto, il sito continuava a
segnalare il falso positivo.

I warnings sono stati corretti solo se effettivamente utili:
\begin{itemize}
\item sono stati introdotti volutamente titoli o link ridondanti per
aiutare l'utente a capire dove si trova;
\item l'uso dei tabindex viene segnalato come warning, anche se questi sono
stati messi per permettere all'utente di raggiungere subito i link della
pagina più rilevanti, saltando elementi di navigazione o breadcrumb a seconda
dei casi.
\end{itemize}

I contrasti dei colori sono stati regolati per riuscire a superare almeno il
livello di accessibilità WCAG 2.0 AA.

\subsection{Vischeck}
Vischek è uno strumento che permette di visualizzare immagini e pagine web
come le vedrebbe una persona che presenta deficit visivi quali:
\begin{itemize}
\item \textbf{Deuteranopia}: una forma di deficit di colore rosso/verde
\item \textbf{Protanopia}: un'altra forma di deficit di colore rosso/verde
\item \textbf{Tritanopia}: una forma di parziale o insufficiente
discriminativa per il blu e il violetto
\end{itemize}
A causa del fatto che lo strumento per analizzare le pagine web era momentaneamente inutilizzabile sono state effettuate delle prove attraverso degli screen delle pagine web
valutando la visione generale che avrebbero avuto utenti affetti dai deficit precedentemente
esposti.

\begin{figure}[H]
\begin{minipage}{0.45\textwidth}
\includegraphics[width=\linewidth]{images/screen/zante.png}
\subcaption{Pagina originale (nessun deficit visivo presente)}
\end{minipage}
\hspace{\fill}
\begin{minipage}{0.45\textwidth}
\includegraphics[width=\linewidth]{images/screen/deuteranope.jpg}
\subcaption{Deuteranopia}
\end{minipage}
\vspace*{0.5cm}
\begin{minipage}{0.45\textwidth}
\includegraphics[width=\linewidth]{images/screen/protanope.jpg}
\subcaption{Protanopia}
\end{minipage}
\hspace{\fill}
\begin{minipage}{0.45\textwidth}
\includegraphics[width=\linewidth]{images/screen/tritanope.jpg}
\subcaption{Tritanopia}
\end{minipage}
\caption{Test con Vischek su una pagina di terzo livello}\label{multiavp}
\end{figure}

Dalle quattro immagini si può vedere come le pagine rimangano comunque accessibili e
in particolare i link sono comunque visibili e ben distinguibili dall'utente.

\subsection{Fangs}\label{sec:fangs} %DONE
\textit{Fangs} è un'estensione per i browser che consente di visualizzare un
sito nello stesso modo in cui verrebbe visualizzato da uno screen reader,
fornendo il testo come sarebbe letto da questo dispositivo, la lista delle
intestazioni e la lista dei link presenti nella pagina.

In sintesi, come principali problemi vengono individuati:
\begin{itemize}
\item Le parole straniere vengono lette in italiano;
\item A volte i numeri sono scritti in numero; ciò vuol dire che se l'utente
ha la lingua di default del computer/browser impostata su inglese (come nel
caso di un verificatore) si hanno frasi come ``\textit{La società distopica in
cui vive Beatrice Prior è suddivisa in five fazioni}", mentre in casi simili
un numero dovrebbe essere scritto per esteso;
\item Si vedono i nomi dei file delle immagini anzichè un testo alternativo;
\item Quando presente, il testo delle immagini spesso trasmette poca
informazione, ridondando semplicemente il titolo anzichè rimanere vuoto
(soluzione preferibile alla ridondanza);
\item Le abbreviazioni non vengono estese, diventando potenzialmente
incomprensibili.
\end{itemize}

\subsection{Lynx}\label{sec:lynx} %DONE
Le pagine del sito sono state anche navigate tramite il browser testuale
\textit{Lynx}.

Come principale difetto, non è possibile effettuare l'accesso a funzioni come
accesso e registrazione, poichè tutti i pulsanti hanno la sottomissione
dell'input gestita tramite JavaScript, quindi per testare tutte le pagine, si è
provveduto a fare una copia dei file una volta loggati tramite browser e poi
ogni pagina è stata aperta con Lynx.

Gli errori rilevati sono gli stessi discussi nelle altre sezioni.

\subsection{Performance}
Per terminare la fase di testing sono state eseguite, con strumenti
automatici, una serie di verifiche riguardo le performance del sito.

Ci siamo avvalsi di tool online\footnote{\texttt{https://developers.google.com/speed/pagespeed/insights/}, \texttt{http://www.webpagetest.org/},
\texttt{http://gtmetrix.com/}} che analizzano in profondità le pagine HTML
fornite e restituiscono: consigli di ottimizzazione, scale di punteggi per ogni
aspetto del sito e una serie di statistiche che vedremo in dettaglio.\footnote{NB: Per poter utilizzare i tester il sito è stato ``hostato" sulla piattaforma GitHub Pages, come scritto all'inizio di questa sezione}

Al fine di uniformare i risultati raccolti da questi strumenti abbiamo deciso
di mostrare solo alcune delle pagine analizzate, una per ogni livello
gerarchico.
In particolare le più significative per ogni livello (\texttt{homepage.html},
\texttt{citta.html},\texttt{london.html}) hanno dato i seguenti esiti:

\begin{figure}[h]
\includegraphics[width=\linewidth]{images/performance/webpagetest/home.png}
\caption{Homepage: tempi di caricamento}
\label{fig:tempiHome}
\end{figure}

\begin{figure}[h]
\includegraphics[width=\linewidth]{images/performance/webpagetest/home-graph.png}
\caption{Homepage: Grafico a 'torta' dei tempi di caricamento}
\end{figure}

In homepage si nota la prevalenza del flusso delle immagini, come evidenziato
in fig.\ref{fig:tempiHome}, dal diagramma circolare delle richieste e della
quantità di dati trasmessi dal server.

Come è facile aspettarsi, le immagini occupano in termini di peso la maggior
parte del sito e le richieste per il loro recupero rappresentano più del 40\%
delle richieste totali.


Lo stesso esito è stato conseguito in \texttt{citta.html} e
\texttt{london.html} (fig.\ref{fig:tempiOther}), sebbene \texttt{citta.html}
sia leggermente più pesante dal momento che viene fatto un uso più intenso dei
fogli di stile.

Dopo aver eseguito i test sulle stesse pagine per 3 volte abbiamo stimato che il tempo medio di caricamento complessivo si aggira intorno ai 1400ms.

I test sulla velocità di caricamento sono stati effettuati su di un client residente negli Stati Uniti avente browser Chrome.


\begin{figure}[h]
\begin{minipage}{0.45\textwidth}
\includegraphics[width=\linewidth]{images/performance/webpagetest/citta.png}
\subcaption{\textit{citta.htm}l: tempi di caricamento}
\end{minipage}
\hspace{\fill}
\vspace*{0.5cm}
\begin{minipage}{0.45\textwidth}

\includegraphics[width=\linewidth]{images/performance/webpagetest/citta-graph.png}
\subcaption{\texttt{citta.html}: grafico a 'torta'}
\end{minipage}

\begin{minipage}{0.45\textwidth}
\vspace*{0.5cm}
\includegraphics[width=\linewidth]{images/performance/webpagetest/london.png}
\subcaption{\texttt{london.html}: tempi di caricamento}
\end{minipage}
\hspace{\fill}
\vspace*{0.5cm}
\begin{minipage}{0.45\textwidth}

\includegraphics[width=\linewidth]{images/performance/webpagetest/london-graph.png}
\subcaption{\textit{london.html}: grafico a 'torta'}
\end{minipage}
\caption{Performance in \texttt{citta.html} e \texttt{london.html}}\label{multiavp}
\label{fig:tempiOther}
\end{figure}

Complessivamente il sito ha totalizzato una media di 71 punti su 100 totali in Google PageSpeed Insights.

Le ottimizzazioni suggerite per aumentarne il punteggio andavano per lo più
contro standard W3C e/o compatibilità con i browser più obsoleti. Abbiamo
quindi deciso di adottare le soluzioni che permettessero di mantenere
l'accessibilità del sito, cercando di alleggerirlo con piccoli accorgimenti
(e.g. minificazione della libreria Require.js, immagini a risoluzione
inferiore mantenendo una certa qualità di fondo, ecc.) trovando un buon
compromesso tra velocità di caricamento e compatibilità con più browser. % 0%

%%%%%%%%%%%% B====>
%PER NOI: OGNI SEZIONE CHE AGGIUNGETE BASTA METTERLA NELLA CARTELLA sections
%con un certo nome (facciamo finta che sia pippo.tex) E
%POI AGGIUNGERE QUI SOTTO \include{sections\pippo} COME HO FATTO CON L'ABSTRACT
%%%%%%%%%%%% B====>

%----------------------------------------------------------------------------------------

\newpage % Start the article content on the second page, remove this if you have a longer abstract that goes onto the second page

\end{document}

%%%%%%%%%%% B====>
% Dovete stare attenti ad avere installato texlive-full e texlive-publisher (o nome simile)
% Per compilare, make da terminale nella folder dove è contenuto questo file (ho già fatto io il makefile)
% Prima di committare, make clean e cancellare *.pdf
%%%%%%%%%%% B====>
