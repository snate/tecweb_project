\section{Accessibilità}
Nel sito non è stato indicato nessuno standard di accessibilità raggiunto.
Con strumenti di validazione, è facile notare che effettivamente non vengono
soddisfatte nè le richieste di WAI-A WCAG2.0 nè della ``section 508".
Di seguito verranno infatti descritte le mancanze rilevate nel sito in termini
di accessibilità, ordinate secondo le 14 linee guida di WAI
%;dove non conforme, verrà indicato la priorità del punto di controllo dove è stata riscontrata la mancanza
.

\subsection{Linee guida per l'accessibilità}
In questa sottosezione viene descritto come il gruppo ha rispettato le linee
guida del WAI.

\subsubsection{Fornire alternative equivalenti al contenuto audio e visivo}
\begin{enumerate}
\item Non sempre viene fornito un equivalente testuale, ad esempio nella
pagina principale mancano gli attributi \texttt{alt} ai tag \texttt{img}
(\textit{Priorità 1});
\item Non sono presenti mappe immagine, perciò 1.2 è rispettato;
\item Non sono presenti audio, perciò 1.3 è rispettato;
\item Non sono presenti filmati, perciò 1.4 è rispettato;
\item Non sono presenti mappe immagine, perciò 1.5 è rispettato;
\end{enumerate}

Il punto di controllo 1.1 fallisce e questo è sufficiente per dire che la
pagina non è accessibile per WAI-A WCAG2.0. Di seguito non verrà più
evidenziato tale fatto (nemmeno per i gradi successivi di accessibilità) poichè
il non-rispetto di un solo punto di controllo compromette il grado complessivo
di accessibilità del sito.

\subsubsection{Non fare affidamento sul solo colore}
\begin{enumerate}
\item L'informazione arriva tramite indicazioni indipendenti dal colore;
\item %CONTRASTO
\end{enumerate}

\subsubsection{Usare marcatori e fogli di stile e farlo in modo appropriato}
\begin{enumerate}
\item Le pagine sono scritte secondo lo standard \textbf{XHTML 1.0 Strict},
sebbene poi non sempre venga rispettato;
\item Tutte le pagine si riferiscono correttamente allo schema \textbf{DTD} di
XHTML 1.0 Strict;
\item Il tag \textit{hr} è utilizzato per meri scopi presentazionali, ovvero
l'inserire un bordo tra delle intestazioni e il contenuto sottostante (ad
esempio in \texttt{about.html}); oltre a questo, sono talvolta inseriti spam per puro scopo presentazionale e non semantico, come quello in \texttt{\#path} (\textit{Priorità 2});
\item Vengono utilizzate delle unità assolute come \texttt{font-size: medium}
per il testo (\textit{Priorità 2});
\item Gli elementi di intestazione sono usati correttamente, tuttavia si
potrebbe pensare di inserirne altri per favorire la lettura del contenuto da
parte di screen-reader o altri dispositivi che fanno molto affidamento sulle
intestazioni;
\item Vi sono alcune \texttt{dl} che vengono scritte e marcate come
\texttt{ul}, comportando l'aggiunta di tag come \texttt{strong} a parti
dell'item listato per emulare i tag \texttt{dt} e \texttt{dd}
(\textit{Priorità 2});
\item Non sono presenti citazioni, perciò 3.7 è rispettato;
\end{enumerate}

\subsubsection{Chiarire l'uso di linguaggi naturali}
\begin{enumerate}
\item Non viene mai segnalato il cambio linguaggio, ad esempio \textit{home} o
\textit{home} in \texttt{accesso.html} (\textit{Priorità 1});
\item Non vengono mai segnalate abbreviazioni, ad esempio \textit{LDV} in
qualsiasi pagina di un libro o ``\textit{J.K. Rowling}" in
\texttt{ricerca.html} (\textit{Priorità 3});
\item Viene correttamente identificato il linguaggio del documento, perciò 4.3
è rispettato.
\end{enumerate}

\subsubsection{Creare tabelle che si trasformino in maniera elegante}
\begin{enumerate}
\item Non sono presenti tabelle, perciò 5.1 è rispettato;
\item Non sono presenti tabelle, perciò 5.2 è rispettato;
\item Non sono presenti tabelle, perciò 5.3 è rispettato;
\item Non sono presenti tabelle, perciò 5.4 è rispettato;
\item Non sono presenti tabelle, perciò 5.5 è rispettato;
\item Non sono presenti tabelle, perciò 5.6 è rispettato.
\end{enumerate}

\subsubsection{Assicurarsi che le pagine che danno spazio a nuove tecnologie
si trasformino in maniera elegante}
\begin{enumerate}
\item Il contenuto, seppur presentando errori di markup, è sufficientemente
leggibile e comprensibile senza fogli di stile, perciò 6.1 si può ritenere
rispettato;
\item Non si ritiene che i contenuti dinamici del sito possano essere
modificati una volta inseriti, se non per i commenti in coda ai libri;
tuttavia l'aggiornamento automatico tramite AJAX o altre tecnologie esulava
dagli obiettivi del corso di Tecnologie Web, perciò 6.2 si può ritenere
rispettato;
\item Le pagine perdono gran parte della loro usabilità disattivando
JavaScript, ad esempio ogni form anzichè avere un input di tipo \textbf{submit}
ha un pulsante normale e la sottomissione dei dati inseriti è completamente
gestita via JavaScript, compromettendo normali operazioni che sarebbero
possibili utilizzando nella struttura i tag semanticamente corretti
(\textit{Priorità 1});
\item I gestori degli eventi non dipendono da particolari dispositivi che
utilizzano JavaScript, perciò 6.4 è rispettato;
\item Le pagine che non risultano accessibili disattivando JavaScript non
presentano alcuna pagina o presentazione alternativa (\textit{Priorità 2}).
\end{enumerate}

\subsubsection{Assicurarsi che l'utente possa tenere sotto controllo i
cambiamenti di contenuto nel corso del tempo}
\begin{enumerate}
\item Nessun componente nella pagina causa fenomeni di sfarfallio, perciò 7.1
è rispettato;
\item Nessun componente nella pagina causa fenomeni di lampeggiamento, perciò
7.2 è rispettato;
\item Nessun componente nella pagina si muove una volta caricato, perciò 7.3 è
rispettato;
\item Le pagine non si aggiornano automaticamente, perciò 7.4 è rispettato;
\item Non vi sono pagine che effettuano auto-reindirizzamento, perciò 7.5 è
rispettato.
\end{enumerate}

\subsubsection{Assicurare l'accessibilità diretta delle interfacce utente
incorporate}
\begin{enumerate}
\item gli script sono ritenuti accessibili verso i dispositivi che li
supportano, perciò 8.1 è rispettato.
\end{enumerate}

\subsubsection{Progettare per garantire l'indipendenza da dispositivo}
\begin{enumerate}
\item Non sono presenti mappe immagine, perciò 9.1 è rispettato;
\item Ogni elemento è accessibile indipendemente dal dispositivo con cui si
naviga nel sito, perciò 9.2 è rispettato;
\item Gli eventi si riferiscono a gesture o actions astratte o logiche e non
relative a specifici dispositivi, perciò 9.3 è rispettato;
\item Non è specificato un ordine di tabulazione diverso da quello di default, sebbene potrebbe essere utile in questo sito (\textit{Priorità 3});
\item Non sono definite scorciatoie da tastiera, il che potrebbe essere
positivo o negativo a seconda delle motivazioni che risiedono dietro questa
scelta; 9.5 si può ritenere rispettato.
\end{enumerate}

%%%%%%%%%%%%CARLO DA QUI NON HO PIÙ SCRITTO NULLA

\subsubsection{Meccanismi di fallback}
Ogniqualvolta che una certa tecnologia non fosse disponibile per visualizzare i
contenuti previsti, si è scelto di non fornire tale contenuto (perché
di secondaria importanza) oppure il sito offre dei meccanismi per i quali la
degradazione è elegante (ad esempio la searchbar, se acquista il focus, si
ingrandisce istantaneamente anzichè effettuare una transizione).

\subsubsection{Raccomandazioni W3C}
Come detto in precedenza, le immagini presentano sempre l'alternativa testuale
inserita utilizzando la tecnologia offerta da W3C.

Non sono presenti formati come shockwave e PDF.

\subsubsection{Orientamento}
Vengono inoltre forniti una mappa del sito ed una pagina di F.A.Q. riferite in
ogni pagina del sito. Oltre a questo, in tutte le pagine eccetto la home (dove
non è ritenuta necessaria) è presente una breadcrumb che indica all'utente la
sua posizione all'interno del sito.

\subsubsection{Navigazione}
I link sono evidenziati in modo che siano distinguibili attraverso un test di \textit{Drue Miller} e forniscono sempre un attributo ``title" che informa
l'utente sul contenuto della destinazione.
In particolare:
\begin{itemize}
\item i link sono in grassetto se presenti nel testo;
\item i colori sono ben distinguibili come spiegato nella sezione \ref{sec:Pres-Colore};
\item i link visitati e non sono sempre riconoscibili gli uni dagli altri, ad
eccezione del caso in cui questi siano nel \textbf{nav} o nel \textbf{footer};
\item altri accorgimenti su come vengono trattati e specializzati gli elementi
della navigazione vengono discussi nella sezione \ref{sec:presentazione} a pagina \pageref{sec:presentazione}.
\end{itemize}

In tutte le pagine eccetto la home (dove la navigazione viene indirizzata
volutamente verso le categorie), è presente un menù di navigazione contenente
i ``sibling" della pagina o, se l'utente si trova in una pagina di una
località, visualizza i riferimenti alla homepage e alle categorie.

In questo elemento viene indicata anche la pagina corrente con un'icona
oppure, qualore l'utente si trovasse in una pagina di III livello, tale icona
sarebbe differente ma si troverebbe comunque a fianco della voce relativa alla
categoria di appartenenza.

\subsubsection{Semplicità dei contenuti}
Il layout è coerente, consistente e riconoscibile in tutte le pagine del sito,
come descritto nelle sezioni \ref{sec:struttura} e \ref{sec:presentazione} ed
allo stesso tempo si è cercato di tenere un linguaggio semplice nei contenuti.

\subsection{Fornire alternative equivalenti al contenuto audio e visivo}
Le pagine del nostro sito rimangono accessibili anche se l'utente non può
utilizzare alcune tecnologie o funzionalità per scelta personale, se il
suo dispositivo di navigazione non le supporta o per cause di forza maggiore
(e.g. svantaggi dal punto di vista fisico e psichico).

\subsubsection{Separazione struttura-comportamento}
In ogni pagina del sito non è stato introdotto alcun attributo per la gestione
degli eventi associata ad un elemento strutturale e gli script sono inclusi
nella pagina e non fanno parte del corpo di questa.
Gli script JavaScript, a loro volta, non si preoccupano di creare intere pagine
dinamicamente, ma si limitano a creare frammenti di queste in modo opportuno,
ovvero quando ciò \textbf{non} è possibile staticamente.

\subsubsection{Separazione struttura-presentazione}
Le pagine del sito non contengono alcun foglio di stile ma ogni foglio di stile
viene incluso in queste. Oltre a ciò, ogni elemento strutturale contiene
solamente attributi relativi al suo significato semantico e non presentazionale
(ad esempio non sono stati utilizzati gli attributi \texttt{bgcolor} e
\texttt{font}).

Ogni id e ogni classe è stata dichiarata utilizzando nomi relativi alla
semantica e non al modo in cui verranno presentati.

\subsubsection{Separazione presentazione-comportamento}
Gli script JavaScript non vanno a modificare le regole con cui gli elementi
sono presentati all'interno delle pagine del sito, ma si limitano ad attribuire
o aggiungere ad essi classi che verranno trattate con dei fogli di stile.

\subsubsection{Porzioni di sito visibili}
L'area cosidetta ``\textit{above the fold}" contiene sempre i contenuti di maggior
rilievo della pagina visualizzata, mentre elementi di minore importanza sono
presenti in parti come il footer che non sempre possono essere visibili all'apertura di una nuova finestra.

