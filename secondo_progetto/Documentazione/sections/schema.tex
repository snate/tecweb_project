\section{Schema organizzativo}\label{sec:schema-org}
Lo schema organizzativo del sito è chiaramente ambiguo:
\begin{itemize}
\item Non ci sono abbastanza informazioni per determinare se un libro nella
sezione ``nuovi" della home sia un libro uscito pubblicamente di recente o un
libro appena arrivato nella collezione della libreria;
\item Nella sezione ``più votati" nella home non è possibile sapere se
l'utente troverà lo stesso libro in tale sezione in due visite compiute in
momenti differiti;
\item Nella sezione ``Generi", un libro potrebbe appartenere a due categorie
diverse (ad esempio ci si aspettava di trovare \textit{Divergent} in
``Narrativa");
\item Nella sezione ``Generi", dopo la selezione del genere non è chiaro
l'ordine in cui sono visualizzati i libri;
\item La ricerca, per come è presentata, farebbe pensare ad uno schema
organizzativo; tuttavia, si parlerà più in dettaglio di ciò nella sezione
seguente ed è doverso notare che non sempre fornisce i risultati attesi (ad
esempio eseguendo la ricerca per autore e immettendo la lettera ``r" viene
mostrato solamente un libro di Glenn Cooper).
\end{itemize}

In particolare, la ricerca per collana è ambigua perchè non vi sono
indicazioni su come svolgere la ricerca: l'utente quindi non sa cosa può
inserire precisamente in tale campo per effettuare la ricerca.

\subsection{Elenchi di libri}\label{sec:schema-elenchi}
Dopo varie ricerche e dopo aver esplorato tutti i generi letterari offerti, ai
verificatori è parso che i libri in questi elenchi siano visualizzati secondo
l'ordine cronologico di apparizione nella libreria, dal più vecchio al
più recente. Tuttavia questa scelta sembra poco plausibile poichè si ritiene
che gli utenti siano interessati alle opere più nuove, come la selezione dei
sei titoli proposta in home.

Per questo motivo si ritiene che manchi del tutto uno schema organizzativo e
che l'ordine trovato nella generazione degli elenchi sia \textbf{incidentale},
mentre sarebbe sufficiente uno schema organizzativo esatto che rispetta
l'ordine alfabetico (o presentando i libri dal più recente al meno recente,
evidenziando la data di pubblicazione/inserimento) per poter aiutare l'utente
nell'esplorazione del sito.